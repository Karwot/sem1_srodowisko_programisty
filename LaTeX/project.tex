\documentclass{report}

\usepackage{polski}
\usepackage[utf8]{inputenc}
\usepackage[unicode]{hyperref}

\title{Projektowanie i tworzenie stron i aplikacji webowych}
\date{06-01-2019}
\author{Kacper Karwot}

\begin{document}
 	\pagenumbering{gobble}
	\maketitle
	\newpage
	\chapter{Wstęp}
	\section{Czym jest web development?}
	Web development to wszystkie zadania, które wchodzą w skład tworzenia strony sieci web, która może być dostępna w Internecie lub intranecie. 
	Web development to szerokie pojęcie, które może obejmować zarówno tworzenie prostej, statycznej strony zawierającej tekst lub 			skomplikowaną aplikację webową, usługę e-commerce lub serwis społecznościowy. 
 Wśród osób pracujących w branży, pojęcie "web development" zazwyczaj odnosi się do nie-projektowych aspektów budowania stron: tworzenia markupu i kodowania. Bardzo często używane są systemy zarządzania treścią(CMS) aby ułatwić dokonywanie zmian i pozwolić osobą mniej zaawansowanym technicznie na pracę nad stroną.
 	\newline
 	W większych organizacjach i biznesach, zespoły web developerskie mogą liczyć nawet setki osób i pracować według takich metodologii jak Agile podczas tworzenia produktu. Przy mniejszych projektach wystarczający może być jeden stały lub kontraktowy pracownik, ewentualnie pomocnicy specjalizujący się w grafice lub technicy IT. 
	\newline
	Wyróżniamy trzy rodzaje specjalizacji web developerów:
	\begin{itemize}
	\item[--] front-end developer,
	\item[--] back-end developer,
	\item[--] full-stack developer
 	\end{itemize}
 	\newpage
	\section{Historia web developmentu}
	
	\newpage 
	\chapter{Web development współcześnie}
	\section{Front-end}
	\newpage
	\section{Back-end}
	\newpage
	\section{Trendy w 2019 i 2020 roku}


	\newpage
	\chapter{Kim jest web developer?}
	\section{Kto może zostać web developerem?}
	\newpage
	\section{Czym zajmuje się web developer}

\tableofcontents
\end{document}